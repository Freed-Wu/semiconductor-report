\documentclass[../main]{subfiles}
\begin{document}

\chapter{优势和局限}%
\label{cha:comparision}

早在 NAND 闪存发明之前, NOR 闪存就已经问世。两者区别如图~\ref{fig:polar}所示
。目前 NOR 闪存因为可以直接执行指令被广泛用于存储 BIOS 。早期的手机上会有 NOR
闪存存储 NAND 闪存的驱动程序,用于之后加载 NAND 闪存。但后来三星率先退出
NORless 技术,将 NAND 闪存的驱动程序写入 CPU 的片上 ROM 里,从而将 NOR 闪存逐
出手机,大幅降低了手机成本。

早期的 NAND 闪存一个存储单元只有 1 位,之后出于节省成本的需要,逐渐出现 2、3
、4 位。目前高端固态硬盘仍为 SLC ,多媒体卡、U 盘以 MLC、TLC 居多。

\begin{figure}[htbp]
  \centering
\begin{tikzpicture}[transform shape]
  \begin{polaraxis}[
      xtick = data,
      xticklabels = {片上执行, 存储难易, 每位平均价, 最小工作电量, 耗电, 写速, 读速, 容量, 片上执行},
      ytick = data,
      yticklabels = {好, 差},
    ]
    \addplot table {data/polar/nor.dat};
    \addplot table {data/polar/nand.dat};
    \legend{NOR, NAND}
  \end{polaraxis}
\end{tikzpicture}
  \caption{闪存比较}%
  \label{fig:polar}
\end{figure}

\begin{table}[htbp]
  \centering
  \csvautobooktabular[respect percent]{tables/bit.csv}
  \caption{每存储单元比特位比较}%
  \label{tab:bit}
\end{table}

\end{document}
