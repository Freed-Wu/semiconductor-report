\documentclass[../main]{subfiles}
\begin{document}

\chapter{当前行业发展和主要厂商}%
\label{cha:development}

\section{当前行业发展}%
\label{sec:current_status}

如图~\ref{fig:bar}, NAND 闪存工艺逐渐往更微小的尺寸发展。各公司之间已经差距
不大。~\footnote{统计图颜色是对应公司徽标的颜色。}

\begin{figure}[htbp]
  \centering
  \begin{tikzpicture}[transform shape]
    \begin{axis}[
        ybar = 0,
        bar width = 0.15cm,
        xtick = data,
        xlabel = 时间/年,
        ylabel = 工艺/nm,
        x tick label style = {/pgf/number format/1000 sep = , rotate = 30, anchor = north},
      ]
      \addplot[fill = blue] table {data/bar/samsung.dat};
      \addplot[fill = gray] table {data/bar/micron.dat};
      \addplot[fill = red] table {data/bar/toshiba.dat};
      \addplot[fill = orange] table {data/bar/sk_hynix.dat};
      \legend{三星, 美光, 铠侠, SK 海力士};
  \end{axis}
\end{tikzpicture}
\caption{闪存工艺}%
\label{fig:bar}
\end{figure}

\section{主要厂商}%
\label{sec:major_supplier}

如图~\ref{fig:pie},三星依旧被称为“闪存王”,而依靠机械硬盘起家的西部数据在转
型固态硬盘后也不容小觑,由东芝存储重组后的铠侠仍有足够的家底。

\begin{figure}[htbp]
  \centering
  \begin{tikzpicture}[transform shape]
    \pie[color = {blue, red, blue!50!black, gray, orange, blue!50}]{%
      34.9/三星电子,
      18.1/铠侠,
      14/西部数据,
      13.5/美光科技,
      10.3/SK 海力士,
      8.7/英特尔
    }
  \end{tikzpicture}
  \caption{市场占有率}%
  \label{fig:pie}
\end{figure}

\end{document}
