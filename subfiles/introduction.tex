\documentclass[../main]{subfiles}
\begin{document}

\chapter{基本概念和工作原理}%
\label{cha:introduction}

\section{基本概念}%
\label{sec:concept}

\begin{definition}[NAND 闪存 (Flash Storage)]
  以块 (Block) 为单位擦除 (Erase) 和以页 (Page) 为单位写入 (Program) 的电子式可清除程序化只读存储器 (EEPROM) 。\cite{wiki:xxx,uefi}
\end{definition}

\begin{description}
  \item[裸片 (die)]如图~\ref{fig:die},单独执行命令、返回状态的最小单位;
  \item[平面 (plane)]如图~\ref{fig:3d},为了增加 NAND 闪存的容量,采用 3D
    堆叠技术后一个裸片可以拥有超过一个平面;
  \item[块]擦除的最小单位;
  \item[页]写入的最小单位。
\end{description}

\begin{figure}[htbp]
  \centering
  \begin{tikzpicture}[transform shape]
    \clip (-1.5, 0) rectangle (12, 30*0.3);
    \foreach \X/\Y/\R/\C in {12/30/0/yellow, 6/30/0.1/brown, 2/5/0.15/gray, 1/1/0.2/green/page} {%
      \foreach \x in {0, \X, ..., 12} {%
        \foreach \y in {0, \Y, ..., 30} {%
          \fill[\C] (\x + \R, {(\y + \R)*0.3}) rectangle (\x + \X - \R, {(\y + \Y - \R)*0.3});
        }
      }
    }
    \foreach \i/\C/\N in {0/yellow/裸片, 1/brown/平面, 2/gray/块, 3/green/页} {%
      \fill[\C] (-1.5, \i) rectangle + (0.5, 0.5) node[midway, right, xshift = 5] {\color{black}\N};
    }
  \end{tikzpicture}
  \caption{NAND 闪存结构}%
  \label{fig:die}
\end{figure}

\newcommand\mystack[1]{%
  \begin{tikzpicture}[transform shape, tdplot_main_coords, scale = 3]
    \begin{scope}[yshift = #1]
      \filldraw[fill = blue!50] (0, 0) + (-180:1)
        \foreach \x in {-180, -90, 0, 90} {
          -- + (\x:1) node at + (\x:1) (08-\x) {}
        } -- cycle;
    \end{scope}

    \draw (0, 0) + (-180:1)
      \foreach \x in {-180, -90, 0} {
        -- + (\x:1) node at + (\x:1) (00-\x) {}
      };
    \filldraw[fill = blue!75] (00--180.center) -- (00--90.center) -- (08--90.center) -- (08--180.center) -- cycle;
    \filldraw[fill = blue!25] (00--90.center) -- (00-0.center) -- (08-0.center) -- (08--90.center) -- cycle;

    \foreach \y in {0, 2, ..., #1} {
      \begin{scope}[yshift = \y]
        \draw (0, 0) + (-180:1)
          \foreach \x in {-180, -90, 0} {
            -- + (\x:1) node at + (\x:1) (0\y-\x) {}
          };
      \end{scope}
    }
  \end{tikzpicture}
}
\begin{figure}[htbp]
  \centering
  \begin{subfigure}[htbp]{0.45\linewidth}
    \centering
    \mystack{2}
    \caption{2D NAND}%
    \label{fig:2d}
  \end{subfigure}
  \quad
  \begin{subfigure}[htbp]{0.45\linewidth}
    \centering
    \mystack{16}
    \caption{3D NAND}%
    \label{fig:3d}
  \end{subfigure}
  \caption{3D 堆叠技术}%
  \label{fig:stack}
\end{figure}

\begin{definition}[擦写 (P/E)]
  NAND 闪存只能将 某一位 1 写入 0,如果需要从 0 反转为 1 需要按块擦除。会先
  将该块标记为废块,再在后台自动回收废块。
  NAND 闪存的寿命由擦写总次数衡量。
\end{definition}

\begin{definition}[闪存传递层 (FTL)]
  如图~\ref{fig:map},为了减少对某一块过于频繁的擦写产生坏块,会通过软件的方
  式实现一个闪存传递层,将逻辑块与物理块一一对应。通过修改 FTL 实现闪存寿命均
  衡。 FTL 由主控芯片实现,是闪存产品的市场竞争力之一。
\end{definition}

\begin{figure}[htbp]
  \centering
  \includegraphics{figures/map.pdf}
  \caption{闪存传递层映射}%
  \label{fig:map}
\end{figure}

\begin{definition}[空闲块 (Over Provision)]
  如图~\ref{fig:op},在标示存储空间外预留的一部分存储空间。作用有:
\end{definition}

\begin{itemize}
  \item 当有坏块时会顶替坏块,保证闪存容量不变;
  \item 存储 FTL 表;
  \item 给废块的回收预留空间;
  \item 抑制写放大现象。
\end{itemize}

\newcommand\myopi{4}
\newcommand\myopj{4}
\begin{figure}[htbp]
  \centering
  \begin{tikzpicture}[transform shape]
    \foreach \i in {0, ..., \myopi}{
      \node at (0.5 + \i, -0.25){CH\i};
    }
    \foreach \j in {0, ..., \myopj}{
      \node at (-0.75, 0.5 + \j){Block \j};
    }
    \filldraw[gray, opacity = 0.25] (0, \myopj) rectangle (\myopi + 1, \myopj + 1);
    \draw[step = 0.25, style = help lines] (0, 0) grid (\myopi + 1, \myopj + 1);
    \draw[step = 1] (0, 0) grid (\myopi + 1, \myopj + 1);
    \draw[decorate, decoration = {brace, amplitude = 5pt}] (\myopi + 1, \myopj + 1) -- (\myopi + 1, \myopj) node [midway, xshift = 20pt]{空闲块};
    \draw[decorate, decoration = {brace, amplitude = 5pt}] (\myopi + 1, \myopj) -- (\myopi + 1, 0) node [midway, xshift = 20pt]{存储块};
  \end{tikzpicture}
  \caption{空闲块}%
  \label{fig:op}
\end{figure}

\begin{definition}[写放大 (Write Amplification)]
  当空闲块数量过少时,每次擦写做废块回收会需要大量时间和更多的擦写次数,加速
  坏块的产生,从而进一步减少空闲块,正反馈直至闪存最终因坏块过量数据无法读写
  而报废。
\end{definition}

\section{工作原理}%
\label{sec:working_principle}

如图~\ref{fig:fet},浮栅金属氧化物场效应管 (FGMOSFET) 是闪存的基本单元。

\begin{description}
  \item[读1]在初始状态下,在控制栅加一个低电压,源级与漏级之间会处于导通状态
    ,存储数据 1。
  \item[写入]在控制栅加一个高电压,从源级将电子吸入浮动栅,由于绝缘层不导电,
    电子被囚禁。即使断电也不能逃逸~\footnote{但时间过久时仍然会有少量电子逃逸
    }。
  \item[读0]这时在控制栅加一个低电压,浮动栅的电子会与低电压抵消。源级与漏级
    之间会处于断开状态,存储数据 0。
  \item[擦除]这时在源级加一个高电压,利用浮动栅与源级的隧道效应~\footnote{量
    子力学证明即使浮动栅与源级不连通,只要存在势能差,电子仍然能穿过屏障}。将
    注入到浮空栅的负电荷吸引到源极,排空浮动栅的电子。回到初始状态。
\end{description}

\begin{figure}[htbp]
  \centering
  \begin{tikzpicture}[transform shape, scale = 3]
    \filldraw[fill = gray] (0, 0) rectangle (3, -0.5) node[midway] {\tiny 衬底};
    \filldraw[fill = green] (0, 0) rectangle (1.2, -0.2) node[midway] {\tiny 源级}
      (1.8, 0) rectangle (3, -0.2) node[midway] {\tiny 漏极}
      (1, 0.5) rectangle + (1, 0.25) node[midway] {\tiny 浮动栅级}
      (1, 1.25) rectangle + (1, 0.25) node[midway] {\tiny 控制栅级};
    \filldraw[fill = black] (1, 0) rectangle + (1, 0.5) node[midway] {\tiny\color{white}绝缘体}
      (1, 0.75) rectangle + (1, 0.5) node[midway] {\tiny\color{white}绝缘体};
  \end{tikzpicture}
  \caption{浮栅金属氧化物场效应管}%
  \label{fig:fet}
\end{figure}

\end{document}
